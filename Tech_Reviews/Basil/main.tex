\documentclass[10pt,draftclsnofoot,onecolumn,journal,compsoc]{IEEEtran}
\usepackage[utf8]{inputenc}
\usepackage[letterpaper, margin=0.75in]{geometry}
\usepackage{caption}

\renewcommand{\linespread}{1.0}
\title{Unusual Objects on Road}
\author{
  \IEEEauthorblockN{Team (Group 32) name: Teaching AutoPilot to Dodge\\ Basil Al Zamil} \\
  \IEEEauthorblockA{CS 461: Senior Capstone Fall 2016 \\ Oregon State University}
}
\date{}

\IEEEtitleabstractindextext{
	\begin{abstract}
		The goal of my technology review is to gather information and become familiar with the technologies that will be used in gathering videos and images, which is the testing surface and input of our project, where we aim to break the algorithms using different objects video taped on the road in different climates and in different geographical areas. In this document, I will be discussing the three sections of my technology review: Cameras, Camera Mount, and Video Taking.
	\end{abstract}
}


\begin{document}

\maketitle
\IEEEdisplaynontitleabstractindextext
\IEEEpeerreviewmaketitle

\newpage

\tableofcontents

\newpage


\section{Camera}

The camera is our main source of input for our research project, since the main goal of the project is to find novel objects to the image recognition software that fails to recognize. Hence, we would ensure that the camera would gather as much data as possible and in different environments that provides different objects. //Moreover, the quality and features of the camera can make a great difference in testing the software. However, The quality and features of the camera are limited to the camera that would be supplied to us. 
\\
We will be using a camera provided by our team member Tanner Fry. We would also use cameras provided by Oregon State University. Also, Dr. Li may be provide us with a camera that are best compatible with the cityscapes software.

\subsection{Types} Infrared cameras would allow us to take videos at night. On the other hand, regular cameras, that would be provided to test the software with, are limited by the day light or by the distance of car's headlights beam at night. 
\\
Different weather conditions can raise a concern when choosing a camera. Water proof camera, that are designed to be mounted on the exterior of cars, can tolerate different wheather conditions such as rain and snow. Weather proof cameras would give a better vision of the road than cameras that are mounted on the dash, since windshields can interfer with the accuracy of image recognition of objects on the road. For example, rain or snow can on the windshield can block or hinder the quality of the video taken. In addition, small obstacles can stick on the windshield and hinder or partly block the lenses from taking accurate images of objects on the road.
\\
Hence, the ideal camera type would be to have infrared cameras that are mounted on the exterior of the vehicle, such as on the fender or on the top of the vehicle.


\subsection{Quality}
The quality of the camera is crucial because it can change the produce or eliminate errors of recognizing objects. Hence, the higher the quality the more accurate our testing would be. 

Camera qualities ranges from 480p, 720p, to 1080p, and higher. Although the camera quality is important, the quality of the camera will be limited to the camera that will be provided to us. However, the best camera would be a high definition camera, such as 1080 pixels and above. These cameras would provide accurate images of the objects, especially when the car is in motion. Moreover, the writing speed has to be compatible with the camera in order for them camera to work probably, which we will be discuss in the Video Taking section.


\section{Mounting}
\subsection{Challenges}
However, there are a couple of differences that may limit our research. First, Tesla’s cars have multiple cameras pointing towards multiple direction of the road, such as to the front of the car and to the rear. However, our camera will be pointing forward only. Second, the cameras on automated driven cars are embedded in cars' body, but ours would be mounted on the windshield, the top of the car, and/or on the fenders. This may cause errors due to weather on the windshield, such as rain or snow on the windshield, or errors in mounting the camera correctly, which may cause issues with recognizing objects that may not exist on the real automated cars. 

Regardless, most of the objects will be in front of the car, rather than in the rear. Hence, the cameras the are pointing to the front of the vehicle are the most important cameras.

\subsection{Different Type of Cameras}
Mounting the camera will depend on the type of camera that will be provided. A regular camera, such as a webcam, would simply be mounted on the windshield, where it can collect videos while the car is in motion. The windshield is advised to be clear at all times in order for the software algorithm to recognize objects without external errors, such as dirty windshield. On the other hand, if the camera provided was a weather proof camera, we can mount the camera on the outside of the vehicle, which would gather a clearer video of the camera. 

The type of the camera would give us the options of mounting the camera on different parts of the vehicle. For example, if the camera was a whether proof and designed to be mounted on the outside of the vehicle, we would rather mount it on the exterior of the vehicle. However, since we are limited with the camera that would be provided to us, we are expecting to have a regular camera that would be mounted on the dash.

\subsection{Multiple Cameras}
Multiple cameras can be used in order to collect data and objects from different angles. Thus, this may provide and additional input for testing and debugging purposes. For example, the software may not recognize an object from a certain camera on a the right angle of the car, but it would from dash camera. Hence, mounting different cameras would provide a great additional surface to test our software with.

Hence, depending on the number of cameras that we can use, we would mount two cameras on the vehicle. One on the side of a car, such as on the fender, and another on the top of the car. This way, when we find a novel object, we can re-test the same object from different angles.


\section{Video Taking}

\subsection{Environments}
Although Oregon would be our main area of collected videos, we will also be collecting data from California and Nevada states. This will provide the software with a different climate and weather conditions to test the software. Also, we will be collecting videos in both country roads and city streets. Videos will be collected through driving in freeways, highways, cities, and in country sides. 
\\
Moreover, we are also reaching out to people in different areas of the country, who would volunteer to mount cameras on their vehicle. That way, we would gather more videos that would have different climates and different objects on the road. Thus, we would have a great amount of input to test our software with, which increases the probability of finding novel objects.

\subsection{Memory}
Since high quality cameras will generate a large amount of data (ideally 1080p and above), we would need a large memory to store the videos recorded.

\captionsetup{justification=centering}
\begin{table}[htp]
\begin{center}
\centering
\begin{tabular}{|p{3cm}||p{3cm}|}
 \hline
 \multicolumn{2}{|c|}{Memory Card Size and Length of Video}\\
 \hline
 Memory Card Size & Hours of Recording\\
 \hline
 8 GB card & 2-3 hours\\
 16 GB card & 4-6 hours\\
32 GB card & 6-12 hours \\
64 GB card & 10-20 hours\\
128 GB card & 20-40 hours\\
 \hline
\end{tabular}
\caption{The table shows the length of the video that can be recorded in a range that is relative to the video quality, from standard to 1080 HD}
\end{center}
\end{table}

Ideally, we would have a memory card of size ranging from 64 GB to 128 GB card. Videos can be stored on Memory Card and in a continuous video taping for 10 to 40 hours. After the memory card is full, the videos can be saved on an external storage, such as a hard drive. If smaller memory card sizes were used, some issues may arise with with collecting data, such as overwriting old videos if the video taker didn’t store the videos on time, such as within 2-3 hours in case a 8 GB memory card was used.

\subsection{Writing speed}
Writing speed is important in order to avoid choppy video taking and and to prevent recording lags, especially if the videos are taken in 1080p and above. Another reason that makes writing speed important is that we usually be video taping objects while the car is traveling in motion. Hence, we need a fast writing speed so we can accurately capture the objects.

Most dash-cams requires at least “Class 6” writing speed for 1080 HD, which is 6 MB/s to 20 MB/s. However, it is recommended to use Class 10 writing speed, which writes in 10 MB/s to 30MB/s. 

Hence we will be using Class 6 as minimum, and an 1080 HD camera, combined with a 128 GB memory card.

\subsection{Laws and Regulations Regrading Video Taking} For law and regulations, the dash camera is legal in the united states by federal law by the First Amendment Right in public places. However, it is not legal on private properties. Hence, the driver should use caution regarding private properties.


\end{document}


\usepackage[utf8]{inputenc}
\usepackage[letterpaper, margin=0.75in]{geometry}

\begin{document}
My role in the project is includes: Camera Setup and Video Recording while driving, Gathering Videos from different resources and parts of the country, and having the video work with the software / algorithms.
\end{document}

\documentclass[10pt,draftclsnofoot,onecolumn,journal,compsoc]{IEEEtran}
\usepackage[utf8]{inputenc}
\usepackage[letterpaper, margin=0.75in]{geometry}
\usepackage{caption}

\renewcommand{\linespread}{1.0}
\title{Unusual Objects on Road}
\author{
  \IEEEauthorblockN{Team (Group 32) name: Teaching AutoPilot to Dodge\\ Basil Al Zamil} \\
  \IEEEauthorblockA{CS 462: Senior Capstone Fall 2016 \\ Oregon State University}
}
\date{}

\IEEEtitleabstractindextext{
    \begin{abstract}
        The goal of my technology review is to gather information and become familiar with the technologies that will be used in gathering videos and images that will be used as the testing surface of our project, where we aim to break the algorithms using different objects video taped on the road in different climates and in different geographical areas. In this document, I will be discussing the three sections of my technology review: Cameras, Camera Mount, and Video Taking.
    \end{abstract}
}


\begin{document}

\maketitle
\IEEEdisplaynontitleabstractindextext
\IEEEpeerreviewmaketitle

\newpage

\tableofcontents

\newpage


\section{Data Collection}
\subsection{Overview}

Since the main goal of the project is to find novel objects to the image recognition software that fails to recognize, capturing video footage through our camera is the main source of input. Thus, the quality of the camera has to be a minimum of 1080 HD. Additionally, we would want to capture novel objects on tape, so taking videos in different types of environment is advised.   
\\

\subsection{Camera Type}
We will be mounting the car on the interior side of the windshield of the car. Thus, providing a standard video of the road. However, we need to make sure that the windshield is clean at all times. That way, we can insure that the camera is not blocked by small object in front of the lens. \\

We considered inferred camera, but we decided to discard the idea them since taking video footage during the day would suffice. We were also thinking about mounting the camera on the exterior of the car, but that was also an overhead that was not necessary to accomplish our goal to find novel object to the object-recognition software.\\

One downfall of mounting the camera on the windshield is that the possibility of the interference with the accuracy of image recognition of objects on the road. For example, rain, snow, or other small objects on the windshield can block or hinder the quality of the video taken. Resulting in hindering or partly blocking the lenses from taking accurate images of objects on the road.

\subsection{Criteria}
The criteria consists of the quality of the camera. The quality of the camera is crucial because it can produce or eliminate errors associated with recognizing objects. Hence, we need the camera to be a minium of 1080 HD, in order for the software to accurately recognize objects. \\

Camera qualities ranges from 480p, 720p, to 1080p, and above. Although the camera quality is important, the quality of the camera will be limited to the camera provided to us. The camera that was provided to us by our client, Dr. Li, was a 1080 HD camera, which is the quality needed to avoid image recognition errors that are produced by the camera, rather than the algorithms. The 1080HD cameras can provide accurate images of objects that will be captured on the road, especially when the car is in motion.\\

Another criteria would be the writing speed of the camera’s memory card. The writing speed of the memory card has to be compatible with the camera, in order for the camera to avoid choppy video storage and to prevent recording lags. An example of choppy video writing would be when the writing speed on the SD card is below 5 MB per second, and the camera capture videos in 3840x2160 (also known as 4K).\\

Another issue to keep in mind is that video taping the objects will take place while the car and the objects are in motion. After we take videos of the objects, we will be extracting images from the videos, which are used as an input to test the image recognition software with. Since the objects are in motion, the writing speed has to compatible, or even above the recommended compatibility, in order to capture accurate images of the objects in motion.\\

Most dash-cams requires at least “Class 6” writing speed for 1080 HD, which is 6 MB/s to 20 MB/s. However, it is recommended to use Class 10 writing speed, which writes in 10 MB/s to 30MB/s. Hence we will be using Class 6 as minimum, and an 1080 HD camera.\\

Thus, we chose our camera to supports up to Class 10 writing speed, and our SD memory to be Class 10.

\subsection{Storage}
Since high quality cameras will generate a large amount of data (ideally 1080p and above), we would need a large memory to store the videos recorded.

\captionsetup{justification=centering}
\begin{table}[htp]
\begin{center}
\centering
\begin{tabular}{|p{3cm}||p{3cm}|}
 \hline
 \multicolumn{2}{|c|}{Memory Card Size and Length of Video}\\
 \hline
 Memory Card Size & Hours of Recording\\
 \hline
 8 GB card & 2-3 hours\\
 16 GB card & 4-6 hours\\
32 GB card & 6-12 hours \\
64 GB card & 10-20 hours\\
128 GB card & 20-40 hours\\
 \hline
\end{tabular}
\caption{The table shows the length of the video that can be recorded in a range that is relative to the video quality, from standard to 1080 HD}
\end{center}
\end{table}

Ideally, we would have a memory card of ranging size of 64 GB to 128 GB card. Videos can be stored on memory card and in a continuous video taping for 10 to 40 hours with the memory cards ranging from 64 GB to 128 GB. When the memory card becomes full, the videos can be saved on an external storage.\\

One concern is when the smaller memory card sizes were used, some issues may arise with with collecting data, such as overwriting old videos. Hence, the video taker is advised to regularly check the free space available in the camera’s memory card, and ensure overwriting does not occur unless the videos collected were stored in the external hard drive.\\

Thus, our client, Dr. Li, advised us to use 128 GB for the camera's SD card, which would store up to 12 hours of video recoding, before is starts to overwrite.

For the hard drive, we would be using a 5 TB hard drive. 5 TB is large enough to store the hours of video footage that we will be taking and gathering from other resources.\\

\subsection{Additional Resources}
Besides our own video taking, we have the option to collect videos from other resources. For example, we can contact Oregon’s police department and ask if we can have some video footage recorded by police vehicles for our research. Additionally, Oregon State Car Club use cameras in their cars, such as on the fenders or dash, and record the road while they drive or race. Hence, they would make a great resource for our data collection.


\section{Methods}
\subsection{Camera Position}
We have a number of options to where we can mount the camera on, such as on the top of the car, on the dash, or on the rear. In automated cars, the front camera is the most important camera. Thus, the default option is to have the camera mounted on the dash.\\

The type of the camera would give us the options of mounting the camera on different parts of the vehicle. For example, if the camera was a whether proof and designed to be mounted on the outside of the vehicle, we would rather mount it on the exterior of the vehicle. However, since we are limited to the camera that would be provided to us, we are expecting to have a regular camera that would be mounted on the dash of the car.\\

However, we are to mount the camera on the windshield of the car, since Dr. Li advised us that mounting the camera on the windshield would sufficient, and other options of mounting the camera, such as on the body, is unnecessary. 



\subsection{Multiple Cameras}
Multiple cameras can be used in order to collect data and objects from different angles. Thus, this may provide and additional input for testing and debugging purposes. For example, the software may not recognize an object from a certain camera that is positioned on a the right angle of the car, but it would recognize the object from the dash camera. Hence, mounting different cameras would provide a great additional surface to test our software with.\\

Depending on the number of cameras that we can use, ideally we would mount two cameras on the vehicle. One on the side of a car, and another on the top of the car. This way, when we find a novel object, we can re-test the same object from different camera that captures the same images from a different angle.\\

Eventually, we were only using one single camera to collect video footage, which was the camera that is mounted on the windshield. 


\subsection{Variables For Experiment}
Although Oregon would be our main area of collecting videos, we will also collect data from California and Nevada states. This will provide the software with a different climates to test the software. Also, we will be collecting videos in both country roads and city streets. Videos will be collected through driving in freeways, highways, cities, and in country sides. Additionally, we are also reaching out to people in different areas of the country, who would volunteer to mount cameras on their vehicle. That way, we would gather more videos that would have different climates and different objects on the road. Thus, we would have a larger variety of inputs to test our software with, which increases the probability of finding novel objects.


\section{Limitation}

\subsection{Law}
The dash camera is legal in the united states by federal law by the First Amendment Right in public areas. However, it is not legal on private properties. The driver should use caution regarding private properties.\\

\subsection{Equipment}
There are a couple of differences between automated driven cars and our experimented data collection method, which may limit our research. First, automated driven cars, such as Tesla’s cars, have multiple cameras pointing towards multiple direction of the road, such as to the front of the car and to the rear. Our camera will be pointing forward only. Regardless, most of the objects will be in front of the car, rather than in the rear. Hence, the cameras the are pointing to the front of the vehicle are the most important cameras.\\

Second, the cameras of automated driven cars are embedded in cars’ body, whereas ours would be mounted on the windshield or on the top of the car. This may cause errors in video taking due to the climate, such as if the lenses are blocked by rain or snow. Another error that may occur is in mounting the camera. The mounting hardware must be tested regularly in order to ensure that the camera is not moving.

\section{Change Log}
    \begin{itemize}
        \item Updated the overview to target the failure of the algorithm, which is the goal of our project.
        \item Deleted the inferred camera information, since we are not using inferred cameras.
        \item Added a downfall of windshield cameras
        \item We thought that high definition camera are better. However, Dr. Li informed us that we only need to have a 1080p camera, and that a higher definition would not make a difference in the accuracy of the object recognition software.
        \item updated the out camera and SD card writing speed, which met the criteria of the tech review research.
        \item updated with the decision of where to mount the camera, which was simply on the windshield of the car.
        \item As for Multiple Cameras, we ended up using one camera.
    \end{itemize}

\end{document}


\usepackage[utf8]{inputenc}
\usepackage[letterpaper, margin=0.75in]{geometry}

\begin{document}
My role in the project is includes: Camera Setup and Video Recording while driving, Gathering Videos from different resources and parts of the country, and having the video work with the software / algorithms.
\end{document}

\documentclass[10pt,draftclsnofoot,onecolumn,journal,compsoc]{IEEEtran}

\usepackage[margin=0.75in]{geometry}
\usepackage{graphicx}
\usepackage{caption}
\usepackage{hyperref}
\usepackage{enumerate}
\usepackage{tabu}
\usepackage[english]{babel}\usepackage[numbers]{natbib}
\usepackage{natbib}
\usepackage{longtable}
\usepackage{pgfgantt}


\renewcommand{\linespread}{1.0}

%design doc
\title{Progress Report \\
Unusual Object on the Road}
\author{
  \IEEEauthorblockN{Team (Group 32) name: Teaching AutoPilot to Dodge\\ Basil Al Zamil, Xilun Guo, and Tanner Fry} \\
  \IEEEauthorblockA{CS 461: Senior Capstone Fall 2016 \\ Oregon State University}
}
\date{}

\IEEEtitleabstractindextext{
	\begin{abstract}
	In this document we discuss our progress on our Capstone project through the Fall term. Information on our current status, progress, things we are working on, improvements, and a retrospective will be conveyed.
	\end{abstract}
}


\begin{document}
% cover page    
    \maketitle
    \IEEEdisplaynontitleabstractindextext
    \IEEEpeerreviewmaketitle

    \newpage
% catalog    
    \tableofcontents
 %\resizebox{\textwidth}{!}
    \newpage
% content   
\section{Introduction}
\subsection{Recap}
The Unusual Objects on the Road project goal is to collect a dataset consisting on images and video frames from a vehicles. 
Based on the Tesla crash in June 2016 in which a person was killed we believe that there are significant problems with image recognition algorithms. 
This stems from findings that the software used in the Tesla vehicle incorrectly registered a Semi-truck as a cloud due to the angle of sunlight reflecting off of it. 
Our goal is to compile a set of images from the perspective of an autonomous car that cause a large portion of image recognition algorithms to fail.\newline
To do this we are going to collect video feeds from dashboard cameras and pull out images from the video feeds at intervals. 
We then will filter out the best images which have test conditions we want. From there we will download individual image recognition algorithms available from the list on the Cityscapes website and test our dataset against each teams software. 
This should give us valuable information on the safety of these types of algorithms and where their faults lie. 
This information, feedback, and data can then be handed back to the algorithm designers to hopefully improve the algorithm and save lives in the future.


\section{Current Progress}
After we have done the problem statement and requirement document, we clearly understand what exactly the project is and what we are required to do research, experiment, and testing. 
Technology review document provided us a stage to do research. 
We selected the most appropriate technology for each specific requirement, and nail down into smaller pieces so that we will be able to complete every single small task smoothly. 
While working on the technology review, we roughly plan on how we are going to apply those technologies onto the future progress.
Working on the design document helps us clear and definite on how to setup hardware and software, how to collect more precise data, and how to test the algorithms with the obtained image and video, which need to be broken into frames.
We applied for purchasing the hardware using for collecting data.
Shortly, we will start recording videos and figure out the way to insert the data into the algorithm.
We currently set up all planning documents, and ready for the real actions. 

%\begin{itemize}
%\item one 
%\item two
%\end{itemize}

\section{Weekly Progress}
\subsection{Week 3}
\begin{itemize}
\item Activities: 
On Week 3, we setup the project's Github account, and we wrote our first draft of the problem statement, and sent it Kevin, and to our client Dr. Li for his signature and approval. 
\item Problems: 
Dr. Li made a few notes for us to make few changes on the problem statement.
\item Solutions: 
We made changes on the problem statement, and sent it to Dr. Li for approval.
\end{itemize}

\subsection{Week 4}
\begin{itemize}
\item Activities:
We continued working on the problem statement. We also met with Dr. Winters to get feedback on our document. In addition, we had our first meeting with our TA Nels.
\item Problems: 
We received conflicting direction from Dr. Winter, Kevin, and Nels, where each told us a different way of how to go about writing the problem statement.
\item Solutions: 
We edited our problem statements in a way that it would appeal to different audience. For audience of a technical background, we wrote in a very technical language. At the same time, we had sections for the same topic where the language used appeals to an audience whose background are not necessarily of a technical one, which would the case for our audience at Expo. 
\end{itemize}

\subsection{Week 5}
\begin{itemize}
\item Activities: We worked on the Requirements Documents and turn in a rough draft. We also met as a group and discussed the IEEE standards and how to implement the standards in our research.
\item Problems: 
We wanted to discuss the IEEE standards with Nels this week, and we had a number of questions for Dr. Li, but we could not meet either of them.
\item Solutions: 
We made changes on the problem statement, and sent it to Dr. Li for approval. We reschedule our meeting Dr. Li for later. 
\end{itemize}

\subsection{Week 6}
\begin{itemize}
\item Activities: 
We received feedback from Nels, Dr. Winters, and Dr. Li on the Requirements Document, and we modified the document accordingly, and according to the IEEE standards.
\item Problems: 
Again, we encountered some troubles following guidelines from different instructors: Kevin, Nels, and Dr. Winters, since they provided different point of views.
\item Solutions: 
We met with Nels, and we were able to combine the different guidelines together, and improved our Requirements Documents accordingly.
\end{itemize}

\subsection{Week 7}
\begin{itemize}
\item Activities: 
We continued to work on the requirements document, and we edited it according to IEEE standards. We assigned each member's focus of the project for our tech review, and spitted responsibilities of the of the project.
\item Problems: 
Nels noted that we did not follow the soft and hard requirements in the IEEE.
\item Solutions: 
We went back to our requirements document the same day and edited according to Nels advise and according to the hard and soft section in the IEEE standards.
\end{itemize}

\subsection{Week 8}
\begin{itemize}
\item Activities: 
Working on the tech review individually. 
Trying to figure out the most appropriate technology, and prepare for the design document by discussing with Nels.
\item Problems: 
Some members faced a hard time finding appropriate technologies for their part of the project, such as hardware setup and testing the algorithms under different climates.
\item Solutions: 
We made more research about what coefficients affect the hardware in bad weather, and we discussed the hardware with Dr.Li.
\end{itemize}

\subsection{Week 9}
\begin{itemize}
\item Activities: 
We made a plan for working on the design document together. We had the second meeting with Dr.Li to discuss the road map for our project, obtaining the algorithms for the object detection from cityscapes, and camera equipment.
\item Problems: 
Misunderstanding about where to find and how to apply the algorithms. 
\item Solutions: 
We discussed the issues with Dr. Li, and he guided us to where to find the algorithms and how to apply them. 
\end{itemize}

\subsection{Week 10}
\begin{itemize}
\item Activities: 
We finished the design document, and met with Nels for next term progresses and plans. We worked on the progress report during the weekend.
\item Problems: 
We were not sure about using or not using the IEEE 1016 for the design document, since our project was a research project.
\item Solutions: 
We met with Dr. Winters, and she advised us to follow a different guideline than IEEE, but we received conflicting instructions from Nels, where he asserted that we should follow the IEEE standards in our research project.
\end{itemize}


\section{Interesting Technologies}
The Cityscapes datasets are already semantically marked up in a style that the Cityscapes team has created.
They have created a python scripting library that handles most image recognition algorithm outputs.
The python scripts analyze the outputs of image recognition algorithms tested by the team, semantically mark and label all of the outputs, correctly ID the recognized objects in a labelID paring, and give a class ID to be associated with the object.
The scripts themselves are broken down into five main parts. 
Viewer scripts are designed to view the images and annotations within the datasets. 
Helper scripts and files are included by other scripts to support functionality. 
Preparation scripts are used to convert the "ground truth" annotation into a format used by the specific algorithms approach. 
Evaluation scripts are for validating of image recognition methods. 
And finally an annotation branch of scripts are used for labeling datasets passed through the code.

\section{Retrospective}

\begin{center}
    \begin{tabular}{ | p{0.3\linewidth} | p{0.3\linewidth} | p{0.3\linewidth} |}
    \hline
     Positives & Deltas & Actions \\ \hline
    Feedback from Kevin was very helpful when talking to him. He is must more personable when talking to him outside of a lecture room setting. & We need more consistent feedback between all of the instructors in the course. We had quite a bit of trouble between talking to different people. Nels gave us different feedback compared to Dr. Winters which created a bit of a rift. & We gave this feedback to Nels and he seemed to agree it was a problem, he will pass this information along to Kevin and Kirsten. \\ \hline
    Dr. Li was very helpful and instructive when we had meetings with him& We need to get a little more regular meeting time setup with Dr. Li. Our schedules were very opposed for the middle of the term making it hard to meet& Next term we should be able to organize things a bit better since we know in advance all of our schedules. \\ \hline
    Dr. Li was open to providing any software or products we needed to start collecting data& We felt like we are behind most groups. Other groups already have code/products being built up, we are just now starting testing and collecting datasets & We have set purchase requests to Dr. Li for a camera, hard drive, and SD cards, they should get here during the break so we can start collecting data. \\ \hline

    
    \hline
    \end{tabular}
\end{center}

      
\newpage
% Remember to use PST-Gantt as Nels suggested for the gantt chart
\newcommand{\firstdayoffallterm}{2016-09-21}      % first day of fall term
\newcommand{\startday}{2016-10-02}                % day groups assigned
\newcommand{\fallprogressreportdue}{2016-12-05}   % finals week of fall term
\newcommand{\alphareleasedue}{2017-02-13}         % week 6 of winter term
\newcommand{\betareleasedue}{2017-03-20}          % finals week of winter term
\newcommand{\winterprogressreportdue}{2017-03-20} % finals week of winter term
\newcommand{\releasedue}{2017-05-15}              % monday prior to tentative expo date
\newcommand{\expoday}{2017-05-19}                 % tentative expo date
\newcommand{\finalreportdue}{2017-06-12}          % finals week of spring term
\newcommand{\lastdayofspringterm}{2017-06-16}     % last day of spring term

\begin{figure}

  % gantt chart: http://mirrors.rit.edu/CTAN/graphics/pgf/contrib/pgfgantt/pgfgantt.pdf
  \begin{ganttchart}[x unit=0.15em, time slot format=isodate, link bulge=4]{\firstdayoffallterm}{\lastdayofspringterm}

    % gantt chart title
    \gantttitlecalendar{year, month=shortname} \\

    % gantt chart bars
    \ganttbar{Capstone}{\startday}{\expoday} \\
    \ganttbar[name=problem]{Problem Statement}{\startday}{2016-10-26} \\
    \ganttbar[name=requirements]{Requirements Document}{2016-10-26}{2016-11-04} \\
    \ganttbar{Algorithm analysis}{2016-10-25}{2016-11-25}\\
    \ganttbar{Design Document}{2016-11-10}{2016-12-04}\\
    \ganttbar[name=hardware]{Camera setup}{2016-12-05}{2016-12-10}\\
    \ganttbar[name=fall]{Fall progress report}{2016-11-15}{2016-12-04}\\
    \ganttbar[name=data]{Data collection}{2016-12-10}{2017-2-15}\\
    \ganttbar[name=rainy]{Data from raining condition}{2016-12-15}{2017-01-15} \\
    \ganttbar[name=software]{Software setup}{2016-12-15}{2016-12-30}\\
    \ganttbar[name=data]{Data collection}{2016-11-15}{2017-2-15}\\
    \ganttbar[name=snowy]{Data from snowing condition}{2016-12-10}{2017-01-25} \\
    \ganttbar[name=sunny]{Data from sunny condition}{2017-01-15}{2017-2-15} \\
    \ganttbar[name=object]{Unusual object testing}{2017-02-15}{2017-3-15}\\
    
    \ganttbar{Beta level release}{2017-02-05}{2017-03-05}\\
    \ganttbar[name=winter]{Winter progress report}{2017-02-01}{2017-03-15}\\
    \ganttbar{Engineer expo}{2017-04-15}{2017-05-25}\\
    \ganttbar[name=spring]{Spring progress report}{2017-05-05}{2017-06-13}
    

    % gantt chart links
    \ganttlink{problem}{requirements}
    \ganttlink{data}{rainy}
    \ganttlink{data}{snowy}
    \ganttlink{data}{sunny}

    %\ganttlink[link mid=0.25]{miningid}{changes}

  \end{ganttchart}

  \caption{Gantt Chart: Timeline of Project Tasks}

\end{figure}

\newpage

\bibliographystyle{IEEEtran}
\bibliography{Sources}
\begin{thebibliography}{12}

\bibitem{Cityscapes Scripts} https://github.com/mcordts/cityscapesScripts.
\textit{Part of the Cityscapes team} Marius Cordts, Mohamed Omran.

\bibitem{Cityscapes} https://www.cityscapes-dataset.com/
\textit{Cityscapes subset of Daimler} 2016 Cityscapes Dataset · by Marius Cordts 

\bibitem{Dr. Pedro Kroger} http://pedrokroger.net/choosing-best-python-ide/
\textit{Choosing the Best Python IDE} Dr. Pedro Kroger

\end{thebibliography}
        
\end{document}